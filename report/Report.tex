\documentclass[12p]{article}
\usepackage[margin=1in, headheight=110pt]{geometry}
\usepackage{amssymb, amsmath, amsfonts, amsthm}
\usepackage{mathpazo}
\usepackage{setspace}
% \usepackage{probsoln}
\usepackage{fancyhdr}
\usepackage{hyperref}
\usepackage{float}
\usepackage{tikz}
\usepackage{enumitem}
\usepackage{graphicx}
\usepackage{listings}
\usepackage{caption}
\usepackage{bookmark}

\def\ojoin{\setbox0=\hbox{$\bowtie$}%
  \rule[-.02ex]{.25em}{.4pt}\llap{\rule[\ht0]{.25em}{.4pt}}}
\def\leftouterjoin{\mathbin{\ojoin\mkern-5.8mu\bowtie}}
\def\rightouterjoin{\mathbin{\bowtie\mkern-5.8mu\ojoin}}
\def\fullouterjoin{\mathbin{\ojoin\mkern-5.8mu\bowtie\mkern-5.8mu\o}}

\pagestyle{fancy}
\lhead{Evan Quan, Irene Chan, Patrick Gharib}
\rhead{SENG 300 - Group Project Iteration 1 - Winter 2018}
\title{\vspace{-6ex}Groupd Project Iteration 1}
\date{\vspace{-12ex}}


\newcommand{\code}[1]{\texttt{#1}}

\setlength{\parindent}{0pt}
\begin{document}
% \maketitle
\thispagestyle{fancy}

\begin{titlepage}
  \begin{center}
    \vspace{1cm}
    \Large{\textbf{University of Calgary}}\\
    \Large{\textbf{SENG 300  - Introduction to Software Engineering}}
    \vfill
    \line(1,0){400}\\[1mm]
    \huge{Group Project Iteration 1}\\
    \large{Finding Declarations and References}\\
    \line(1,0){400}\\
    \Large March 14, 2018\\
    \vfill
    \large{Evan Quan, Irene Chan, Patrick Gharib}\\
  \end{center}
\end{titlepage}

% \tableofcontents
% \thispagestyle{empty}
% \clearpage
%
% \onehalfspacing
%
% \setcounter{page}{1}

\section{Structural Diagram}
\begin{figure}[H]
  \includegraphics[width=0.95\textwidth]{SengUml.pdf}
  \caption{} % TODO caption
  \label{fig:structural}
\end{figure}


\section{Sequence Diagrams}
\begin{figure}[H]
  \includegraphics[width=0.95\textwidth]{mainTypeFinder.pdf}
  \caption{} % TODO caption
  \label{fig:sequence1}
\end{figure}

\begin{figure}
  \includegraphics[width=1.0\textwidth]{initFrinder.pdf}
  \caption{} % TODO caption
  \label{fig:sequence2}
\end{figure}

\begin{figure}
  \includegraphics[width=1.0\textwidth]{getAllJavaFilesToString.pdf}
  \caption{} % TODO caption
  \label{fig:sequence2}
\end{figure}

\newpage

\section{State Diagram}
\begin{figure}[H]
  \includegraphics[width=1.0\textwidth]{State_diagram.pdf}
  \caption{} % TODO caption
  \label{fig:state}
\end{figure}

\section{Explanation}

The inner workings of the getConfiguredASTParser() method is non-essential to the client's understanding of the software. Thus the details were abstracted away and not expanded on. Another aspect of the code that was abstracted away was most of the methods in JavaFileReader. We modeled a very broad overview, this informs anyone viewing the model (provided some basic knowledge of java) of what is happening. Had we made our diagram anymore specific in this area, it would draw attention away from the more important functionality of the software, and would only overwhelm the viewer.

\end{document}
